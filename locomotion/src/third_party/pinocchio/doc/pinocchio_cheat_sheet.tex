%%%%%%%%%%%%%%%%%%%%%%%%%%%%%%%%%%%%%%%%%%%%%%%%%%%%%%%
% Pinocchio Cheat Sheet
%%%%%%%%%%%%%%%%%%%%%%%%%%%%%%%%%%%%%%%%%%%%%%%%%%%%%%%

\documentclass[10.5pt]{article}
\usepackage[landscape,a3paper]{geometry}
\usepackage[table]{xcolor}
\usepackage{url}
\usepackage{multicol}
\usepackage{amsmath}
\usepackage{amsfonts}
\usepackage{tikz}
\usetikzlibrary{decorations.pathmorphing}
\usepackage{amsmath,amssymb}

\usepackage{colortbl}
\usepackage{mathtools}
\usepackage{amsmath,amssymb}
\usepackage{enumitem}
\usepackage{arydshln}
\usepackage{wrapfig}

\title{Pinocchio Cheat Sheet}
\usepackage[english]{babel}
\usepackage[utf8]{inputenc}

\usepackage{minted}
\usemintedstyle{vs}


\advance\topmargin-3cm
\advance\textheight3in
\advance\textwidth5in
\advance\oddsidemargin-6cm
%\advance\evensidemargin-1.5in
\parindent0pt
\parskip2pt
\newcommand{\hr}{\centerline{\rule{3.5in}{1pt}}}
%\colorbox[HTML]{e4e4e4}{\makebox[\textwidth-2\fboxsep][l]{texto}
\begin{document}

\begin{center}{\huge{\textbf{Pinocchio Cheat Sheet}}}
\end{center}
\begin{multicols*}{3}
\vspace{-1cm}

\tikzstyle{mybox} = [draw=black, fill=white, very thick,
    rectangle, rounded corners, inner sep=10pt, inner ysep=10pt]
\tikzstyle{fancytitle} =[fill=black, text=white, font=\bfseries]

%------------ Begin Get started ---------------------
\begin{tikzpicture}
\node [mybox] (box){%
    \begin{minipage}{0.9\linewidth}
    \rowcolors{1}{white}{gray!20}
        \begin{tabular}{p{3cm} | p{8.5cm}}
            easy install & \texttt{conda install -c conda-forge pinocchio}\\
            import & \texttt{import pinocchio as pin}\\
            & \texttt{from pinocchio.utils import *} \\
            \textbf{documentation} & \texttt{pin.Model?} \\
        \end{tabular}
    
    \end{minipage}
};

\node[fancytitle] at (box.north) {Get started};
\end{tikzpicture}
%------------ End Get started ---------------------

%------------ Begin Spatial quantities ---------------------

\begin{tikzpicture}
\node [mybox] (box){%
    \begin{minipage}{0.9\linewidth}
        \rowcolors{1}{white}{gray!20}
        \centerline{\textbf{Transforms}}\vspace{0.1cm}
        \begin{tabular}{p{4cm} | p{7.5cm}}
            SE3 & \texttt{aMb = pin.SE3(aRb,apb)} \\ \hline
            \quad unit transformation & \texttt{M = pin.SE3(1)} or \texttt{pin.SE3.Identity()} \\ 
            \quad random transformation & \texttt{pin.SE3.Random()} \\ 
            \quad rotation matrix & \texttt{M.rotation} \\
            \quad translation vector & \texttt{M.translation} \\ \hline
            SE3 inverse & \texttt{bMa = aMb.inverse()} \\
            SE3 action & \texttt{aMc = aMb * bMc} \\
            action matrix & \texttt{aXb = aMb.action} \\
            homegeneous matrix & \texttt{aHb = aMb.homogeneous} \\
            log operation SE3 $\rightarrow$ 6D & \texttt{pin.log(M)} \\
            exp operation & \texttt{pin.exp(M)} \\
        \end{tabular} \newline

        \centerline{\textbf{Spatial Velocity}}\vspace{0.1cm}
        \begin{tabular}{p{4cm} | p{7.5cm}}
            Motion & \texttt{m = pin.Motion(v,w)} \\ \hline
            \quad linear acceleration & \texttt{m.linear} \\
            \quad angular acceleration & \texttt{m.angular} \\ \hline
            SE3 action & \texttt{v\_a = aMb * v\_b} \\
        \end{tabular} \newline
        
        \centerline{\textbf{Spatial Acceleration}}\vspace{0.1cm}
        \begin{tabular}{p{4cm} | p{7.5cm}}
            used in algorithms  & \texttt{a = ($\dot{\omega}$,$\dot{v}_O$)} \\ 
            get classical acceleration & \texttt{a' = a + $(0, \omega \times v_O)$} \\
                          & \texttt{pin.classicAcceleration(v,a, [aMb])} \\
        \end{tabular} \newline

        \centerline{\textbf{Spatial Force}}\vspace{0.1cm}
        \begin{tabular}{p{4cm} | p{7.5cm}}
            Force & \texttt{f = pin.Force(l,n)} \\ \hline
            \quad linear force & \texttt{f.linear} \\
            \quad torque & \texttt{f.angular} \\ \hline
            SE3 action & \texttt{f\_a = aMb * f\_b} \\
        \end{tabular} \newline

        \centerline{\textbf{Spatial Inertia}}\vspace{0.1cm}
       \begin{tabular}{p{4cm} | p{7.5cm}}	
            Inertia & \texttt{Y = pin.Inertia(mass,com,I)} \\ \hline
            \quad mass & \texttt{Y.mass} \\
            \quad center of mass pos. & \texttt{Y.lever} \\
            \quad rotational inertia & \texttt{Y.inertia} \\
        \end{tabular} \newline
        
        \centerline{\textbf{Geometry}}\vspace{0.1cm}
        \begin{tabular}{p{4cm} | p{7.5cm}}
            Quaternion & \texttt{quat = pin.Quaternion(R)} \\
            Angle Axis & \texttt{aa = pin.AngleAxis(angle,axis)} \\ 
        \end{tabular} \newline
        
        \centerline{\textbf{Useful converters}} 
        \vspace{0.2cm}
        \begin{tabular}{p{4cm} | p{7.5cm}}
        	  SE3 $\rightarrow$ (x,y,z,quat) & \texttt{pin.se3ToXYZQUAT(M)} \\
        	  (x,y,z,quat) $\rightarrow$ SE3 & \texttt{pin.XYZQUATToSE3(vec)} \\
        \end{tabular}

    \end{minipage}
};

\node[fancytitle] at (box.north) {Spatial quantities};
\end{tikzpicture}
%------------ End Spatial quantities ---------------------

%------------ Begin Data ---------------------------------
\begin{tikzpicture}
\node [mybox] (box){%
    \begin{minipage}{0.9\linewidth}
        \rowcolors{1}{white}{gray!20}
        \begin{tabular}{p{4cm} | p{7.5cm}}
            Data {\small related to the model} & \texttt{data = pin.Data(model)} \\ 
             & \texttt{data = model.createData()} \\ \hline
            \hspace{0.5em} joint data & \texttt{data.joints} \\
            \hspace{0.5em} joint/[frame] placements & \texttt{data.oMi} /[\texttt{data.oMf}] \\
            \hspace{0.5em} joint velocities & \texttt{data.v} \\
            \hspace{0.5em} joint accelerations & \texttt{data.a} \\
            \hspace{0.5em} joint forces & \texttt{data.f} \\
            \hspace{0.5em} mass matrix & \texttt{data.M} \\
            \hspace{0.5em} non linear effects & \texttt{data.nle} \\
            \hspace{0.5em} centroidal momentum & \texttt{data.hg} \\
            \hspace{0.5em} centroidal matrix & \texttt{data.Ag} \\
            \hspace{0.5em} centroidal inertia & \texttt{data.Ig} \\
        \end{tabular} 
        
    \end{minipage}
};

\node[fancytitle] at (box.north) {Data};
\end{tikzpicture}
%------------ End Data ---------------------------------

%------------ Begin Model --------------------------------
% Some items in this box here seems kind of self-explanatory, good idea to make award of the different attributes and methods but maybe not to best representation (model name - model.name) , others are quite useful.
\begin{tikzpicture}
\node [mybox] (box){%
    \begin{minipage}{0.9\linewidth}
        \rowcolors{1}{white}{gray!20}
        \begin{tabular}{p{4cm} | p{7.5cm}}
            Model {\small of the kinematic tree} & \texttt{model = pin.Model()} \\ \hline
            \quad model name & \texttt{model.name} \\
            \quad joint names & \texttt{model.names} \\
            \quad joint models & \texttt{model.joints} \\
            \quad joint placements & \texttt{model.placements} \\
            \quad link inertias & \texttt{model.inertias} \\
            \quad frames & \texttt{model.frames} \\ 
            \quad \# position variables & \texttt{model.nq} \\ 
            \quad \# velocity variables & \texttt{model.nv} \\ 
            Methods & use ? to get doc and input arguments  \\ \hline
            % Inputs missing, but there are a lot. maybe enough like this and doc should be ask. maybe another hint to the doc ?
            \quad add joint & \texttt{model.addJoint} \\
            \quad append body & \texttt{model.appendBodyToJoint} \\
            \quad add frame & \texttt{model.addFrame} \\
            \quad append child into parent model & \texttt{model.appendModel} \\
            \quad build reduced body & \texttt{model.buildReducedModel} \\
            % findCommonAncestor
        \end{tabular} 

    \end{minipage}
};

\node[fancytitle] at (box.north) {Model};
\end{tikzpicture}
%------------ End Model --------------------------------

%------------ Begin Parsers ----------------------------
\begin{tikzpicture}
\node [mybox] (box){%
    \begin{minipage}{0.9\linewidth}
        \rowcolors{1}{white}{gray!20}
         \begin{tabular}{p{3cm} | p{8.5cm}}
            load a URDF file & \texttt{pin.buildModelFromUrdf(filename,[root\_joint])} \\
            load a SDF file & \texttt{pin.buildModelFromSdf(filename,[root\_joint], root\_link\_name,parent\_guidance)} \\
        \end{tabular} 
        
    \end{minipage}
};

\node[fancytitle] at (box.north) {Parsers};
\end{tikzpicture}
%------------ End Parsers --------------------------------

%------------ Begin Reference Frames ---------------------------------
\begin{tikzpicture}
\node [mybox] (box){%
    \begin{minipage}{0.25\linewidth}
    \includegraphics[]{local_world_aligned.pdf}
    \end{minipage}%
    \begin{minipage}{0.65\linewidth}
        \small{Coordinate system (CS)}\newline
        \\
        \rowcolors{1}{white}{gray!20}
        \begin{tabular}{p{4.0cm} | p{4.0cm}}
            WORLD & world CS \\
            LOCAL & local CS of the joint \\
            LOCAL\_WORLD\_ALIGNED & local CS aligned with WORLD axis \\
            
        \end{tabular} 

    \end{minipage}
};

\node[fancytitle] at (box.north) {Reference Frames};
\end{tikzpicture}
%------------ End Reference Frames ---------------------------------

%------------ Begin Frames ----------------------------
\begin{tikzpicture}
\node [mybox] (box){%
    \begin{minipage}{0.9\linewidth}
    \rowcolors{1}{white}{gray!20}
        \begin{tabular}{p{4cm} | p{7.5cm}}
            placement of all operational frames & \texttt{pin.updateFramePlacements(model, data)} \\ 
            current frame placements wrt origin & \texttt{data.oMf} \\
            frame veloctiy & \texttt{pin.getFrameVelocity(model, data, frame\_id, ref\_frame)} \\ 
            frame acceleration & \texttt{pin.getFrameAcceleration(model, data, frame\_id, ref\_frame)} \\
            frame acceleration & \texttt{pin.getFrameClassicalAcceleration(\newline model, data, frame\_id, ref\_frame)} \\
            frames placement & \texttt{pin.framesForwardKinematics(model, data, q)} \\
            frame jacobian & \texttt{pin.computeFrameJacobian(model, data, q, frame\_id, ref\_frame)} \\ \hline
            frame jacobian time variation & \texttt{pin.frameJacobianTimeVariation(model, data, q, v, frame\_id, ref\_frame)} \\ 
            % maybe better to use symbols ? 
            partial derivatives of the spatial velocity & \texttt{pin.getFrameVelocityDerivatives(model, data, frame\_id, ref\_frame)} \\ 
            partial derivatives of the spatial velocity & \texttt{pin.getFrameVelocityDerivatives(model, data, joint\_id, placement ref\_frame)} \\ 
            partial derivatives of the spatial acceleration & \texttt{pin.getFrameVelocityDerivatives(model, data, frame\_id, ref\_frame)} \\ 
            partial derivatives of the spatial acceleration & \texttt{pin.getFrameAccelerationDerivatives\newline (model, data, joint\_id, placement ref\_frame)} \\ 
        \end{tabular} 
        
    \end{minipage}
};

\node[fancytitle] at (box.north) {Frames};
\end{tikzpicture}
%------------ End Frames ------------------------


%------------ Begin Configuration -------------------------
\begin{tikzpicture}
\node [mybox] (box){%
    \begin{minipage}{0.9\linewidth}
        \rowcolors{1}{white}{gray!20}
        \begin{tabular}{p{4cm} | p{7.5cm}}
            random configuration & \texttt{pin.randomConfiguration(model, [lower\_bound, upper\_bound])} \\
            neutral configuration & \texttt{pin.neutral(model)} \\
            normalized configuration & \texttt{pin.normalize(model, q)} \\
            difference configurations & \texttt{pin.difference(model, q1, q2)} \\
            distance configurations & \texttt{pin.distance(model, q1, q2)} \\
            squared distance configurations & \texttt{pin.squareDistance(model, q1, q2)} \\
            interpolate configuration & \texttt{pin.interpolate(model, q1, q2, alpha)} \\
            integrate configuration & \texttt{pin.integrate(model, q, v)} \\ \hline
            partial derivatives of difference & \texttt{pin.dDifference(model, q1, q2, [arg\_pos])} \\
            partial derivatives of integration & \texttt{pin.dIntegrate(model, q, v, [arg\_pos])}
        \end{tabular} 
        
    \end{minipage}
};
\node[fancytitle] at (box.north) {Configuration};
\end{tikzpicture}
%------------ End Configuration  ---------------------------

%------------ Begin Collision -------------------------
\begin{tikzpicture}
\node [mybox] (box){%
    \begin{minipage}{0.9\linewidth}
        \rowcolors{1}{white}{gray!20}
        \begin{tabular}{p{3cm} | p{8.5cm}}
            placement collision obj & \texttt{pin.updateGeometryPlacements(model, data, geometry\_model, geometry\_data, [q])} \\ 
            collisions detection for all pairs & \texttt{pin.computeCollisions(model, data, geometry\_model, geometry\_data, q)} \\
            collisions detection for a pair &
            \texttt{pin.computeCollisions(geometry\_model, geometry\_data, pair\_index)} \\
            distance from collision & \texttt{pin.computeDistance(geometry\_model, geometry\_data, [pair\_index])} \\
            distance from collision each pair &
            \texttt{pin.computeDistances([model, data], geometry\_model, geometry\_data, [q])} \\
            geometry volume radius & \texttt{pin.computeBodyRadius(model, geometry\_model, geometry\_data} \\ \hline
            BroadPhase &
            \texttt{pin.computeCollisions(broadphase\_manager, callback)} \\
             & \texttt{pin.computeCollisions(broadphase\_manager, stop\_at\_first\_collision)} \\
            + forward kinmetatics to update geometry placements & \texttt{pin.computeCollisions(model, data, broadphase\_manager, q, stop\_at\_first\_collision)} \\
        \end{tabular} 
        
    \end{minipage}
};
\node[fancytitle] at (box.north) {Collision};
\end{tikzpicture}
%------------ End Collision  ---------------------------

%------------ Begin COM ---------------------------------
\begin{tikzpicture}
\node [mybox] (box){%
    \begin{minipage}{0.9\linewidth}
        \rowcolors{1}{white}{gray!20}
        \begin{tabular}{p{4cm} | p{7.5cm}}
            total mass of model & \texttt{pin.computeTotalMass(model, [data])} \\ 
            mass of each subtree & \texttt{pin.computeSubtreeMasses(model, data)} \\ 
            center of mass (COM) & \texttt{pin.centerOfMass(model, data, q, [v, a],[compute\_subtree\_com])} \\ 
            Jacobian COM & \texttt{pin.jacobianCenterOfMass(model, data, [q],[compute\_subtree\_com])} \\ 
        \end{tabular} 
        
    \end{minipage}
};

\node[fancytitle] at (box.north) {Center of Mass};
\end{tikzpicture}
%------------ End COM ---------------------------------

%------------ Begin Energy -------------------------------
\begin{tikzpicture}
\node [mybox] (box){%
    \begin{minipage}{0.9\linewidth}
        \rowcolors{1}{white}{gray!20}
        \begin{tabular}{p{3cm} | p{8.5cm}}
            FK and kinetic Energy & \texttt{pin.computeKineticEnergy(model, data, [q, v]) } \\
            FK and potential Energy & \texttt{pin.computePotentialEnergy(model, data, [q, v]) } \\
            FK and mechanical Energy & \texttt{pin.computeMechanicalEnergy(model, data, [q, v]) } \\
        \end{tabular} 
        
    \end{minipage}
};

\node[fancytitle] at (box.north) {Energy};
\end{tikzpicture}
%------------ End Energy ---------------------------------

%------------ Begin Kinematics -------------------------
\begin{tikzpicture}
\node [mybox] (box){%
    \begin{minipage}{0.9\linewidth}
        \rowcolors{1}{white}{gray!20}
        \begin{tabular}{p{3cm} | p{8.5cm}}
            forward kinematics (FK) & \texttt{pin.forwardKinematics(model, data, q, [v,[a]])} \\ \hline
            FK derivatives & \texttt{pin.computeForwardKinematicsDerivatives( \newline model, data, q, v, a)} \\
            $\left[ \frac{\partial{v}}{\partial{q}},\frac{\partial{v}}{\partial{\dot{q}}} \right]^{WORLD}$ & \texttt{pin.getJointVelocityDerivatives(model, data, joint\_id,pin.ReferenceFrame.WORLD)}   \\
            $\left[ \frac{\partial{v}}{\partial{q}},\frac{\partial{a}}{\partial{q}}, \frac{\partial{a}}{\partial{\dot{q}}}\right]^{LOCAL}$ & \texttt{pin.getJointAccelerationDerivatives(model, data, joint\_id,pin.ReferenceFrame.LOCAL)} \\ 
            % getVelocity, getAcceleration, getClassicalAcceleration
        \end{tabular} 
        
    \end{minipage}
};
\node[fancytitle] at (box.north) {Kinematics};
\end{tikzpicture}
%------------ End Kinematics  ---------------------------

%------------ Begin Jacobian ----------------------------
\begin{tikzpicture}
\node [mybox] (box){%
    \begin{minipage}{0.9\linewidth}
        \rowcolors{1}{white}{gray!20}
        \begin{tabular}{p{3cm} | p{8.5cm}}
            full model Jacobian $\rightarrow$ data.J & \texttt{pin.computeJointJacobians(model, data, [q])} \\
            joint Jacobian & \texttt{pin.getJointJacobian(model, data, joint\_id, ref\_frame)} \\ \hline
            full model dJ/dt & \texttt{pin.computeJointJacobiansTimeVariation(model, data, q, v)} \\
            joint dJ/dt & \texttt{pin.getJointJacobianTimeVariation(model, data, joint\_id, ref\_frame)} \\
        \end{tabular} 
        
    \end{minipage}
};

\node[fancytitle] at (box.north) {Jacobian};
\end{tikzpicture}
%------------ End Jacobian  ------------------------------

%------------ Begin Forward Dynamics ---------------------
\begin{tikzpicture}
\node [mybox] (box){%
    \begin{minipage}{0.9\linewidth}
        \rowcolors{1}{white}{gray!20}
        \begin{tabular}{p{3cm} | p{8.5cm}}
            Articulated-Body Algorithm $\ddot{q}$ & \texttt{pin.aba(model, data, q, v, tau, [f\_ext]) } \\
            Joint Space Inertia Matrix Inv & \texttt{pin.computeMinverse(model, data, [q]) } \\ \hline
            Composite Rigid-Body Algorithm & \texttt{pin.crba(model, data, q)}
        \end{tabular} 
        
    \end{minipage}
};

\node[fancytitle] at (box.north) {Forward Dynamics};
\end{tikzpicture}
%------------ End Forward Dynamics -----------------------

%------------ Begin Inverse Dynamics ---------------------
\begin{tikzpicture}
\node [mybox] (box){%
    \begin{minipage}{0.9\linewidth}
        \rowcolors{1}{white}{gray!20}
        \begin{tabular}{p{3cm} | p{8.5cm}}
            Recursive Newton-Euler Algorithm & \texttt{pin.rnea(model, data, q, v, a, [f\_ext]) } \\
            generalized gravity & \texttt{pin.computeGeneralizedGravity(model, data, q) } \\
            \hline
            dtau\_dq, dtau\_dv, dtau\_da & \texttt{pin.computeRNEADerivatives(model, data, q, v, a, [f\_ext]) } \\
        % nonLinearEffects, computeGeneralizedGravity, computeStaticTorque, computeCoriolisMatrix
        \end{tabular} 
        
    \end{minipage}
};

\node[fancytitle] at (box.north) {Inverse Dynamics};
\end{tikzpicture}
%------------ End Inverse Dynamics -----------------------

%------------ Begin Centroidal ---------------------------
\begin{tikzpicture}
\node [mybox] (box){%
    \begin{minipage}{0.9\linewidth}
        \rowcolors{1}{white}{gray!20}
        \begin{tabular}{p{3cm} | p{8.5cm}}
            Centroidal momentum & \texttt{pin.computeCentroidalMomentum(model, data, [q, v]) } \\ \hline
            Centroidal momentum + time derivatives & \texttt{pin.computeCentroidalMomentumTimeVariation(\newline model, data, [q, v, a]) } \\
        % also mention ccrba, dccrba, computeCentroidalMap, computeCentroidalMapTimeVariation?
        \end{tabular} 
        
    \end{minipage}
};

\node[fancytitle] at (box.north) {Centroidal};
\end{tikzpicture}
%------------ End Centroidal -----------------------------


%------------ Begin General ------------------------------
\begin{tikzpicture}
\node [mybox] (box){%
    \begin{minipage}{0.9\linewidth}
        \rowcolors{1}{white}{gray!20}
        \begin{tabular}{p{3cm} | p{8.5cm}}
            all terms (check doc) & \texttt{pin.computeAllTerms(model, data, q, v) } \\
        \end{tabular} 
        
    \end{minipage}
};

\node[fancytitle] at (box.north) {General};
\end{tikzpicture}
%------------ End General -------------------------------

%------------ Begin Kinematic Regressor -------------------------
\begin{tikzpicture}
\node [mybox] (box){%
    \begin{minipage}{0.9\linewidth}
        \rowcolors{1}{white}{gray!20}
        \begin{tabular}{p{3cm} | p{8.5cm}}
            kinematic regressor & \texttt{pin.computeJointKinematicRegressor(model, data, joint\_id, ref\_frame, [placement])} \\ 
            kinematic regressor & \texttt{pin.computeFrameKinematicRegressor(model, data, frame\_id, ref\_frame)} \\ 
        \end{tabular} 
        
    \end{minipage}
};
\node[fancytitle] at (box.north) {Kinematic Regressor};
\end{tikzpicture}
%------------ End Kinematic Regressor  ---------------------------

%------------ Begin Regressor -------------------------
\begin{tikzpicture}
\node [mybox] (box){%
    \begin{minipage}{0.9\linewidth}
        \rowcolors{1}{white}{gray!20}
        \begin{tabular}{p{3cm} | p{8.5cm}}
            static regressor & \texttt{pin.computeStaticRegressor(model, data, q)} \\ 
            body regressor & \texttt{pin.bodyRegressor(velocity, acceleration)} \\ 
            body attached to joint regressor & \texttt{pin.jointBodyRegressor(model, data, joint\_id)} \\ 
            body attached to frame regressor & \texttt{pin.frameBodyRegressor(model, data, frame\_id)} \\ 
            joint torque regressor & \texttt{pin.computeJointTorqueRegressor(model, data, q, v, a)} \\ 
        \end{tabular} 
        
    \end{minipage}
};
\node[fancytitle] at (box.north) {Regressor};
\end{tikzpicture}
%------------ End Regressor  ---------------------------



%------------ Begin Contact Jacobian -------------------------
\begin{tikzpicture}
\node [mybox] (box){%
    \begin{minipage}{0.9\linewidth}
        \rowcolors{1}{white}{gray!20}
        \begin{tabular}{p{3cm} | p{8.5cm}}
            kinematic Jacobian of constraint model & \texttt{pin.getConstraintJacobian(model, data, contact\_model, contact\_data)} \\ 
            kinematic Jacobian of set of constraint models& \texttt{pin.getConstraintJacobian(model, data, contact\_models, contact\_datas)} \\ 
        \end{tabular} 
        
    \end{minipage}
};
\node[fancytitle] at (box.north) {Contact Jacobian};
\end{tikzpicture}
%------------ End Contact Jacobian  ---------------------------

%------------ Begin Contact Dynamics -------------------------
\begin{tikzpicture}
\node [mybox] (box){%
    \begin{minipage}{0.9\linewidth}
        \rowcolors{1}{white}{gray!20}
        \begin{tabular}{p{3cm} | p{8.5cm}}
            constrained dynamics with contacts & \texttt{pin.forwardDynamics(model, data, [q, v,] tau, constraint\_jacobian, constraint\_drift, damping)} \\ 
            impact dynamics with contacts & \texttt{pin.impulseDynamics(model, data, [q,] v\_before, constraint\_jacobian, restitution\_coefficient, damping)} \\ 
            inverse of the constraint matrix & \texttt{pin.computeKKTContactDynamicMatrixInverse(\newline model, data, q, constraint\_jac, damping)} \\ 
            % getKKTContactDynamicMatrixInverse
        \end{tabular} 
        
    \end{minipage}
};
\node[fancytitle] at (box.north) {Contact Dynamics};
\end{tikzpicture}
%------------ End Contact Dynamics  ---------------------------

%------------ Begin Constraint Dynamics -------------------------
\begin{tikzpicture}
\node [mybox] (box){%
    \begin{minipage}{0.9\linewidth}
        \rowcolors{1}{white}{gray!20}
        \begin{tabular}{p{3cm} | p{8.5cm}}
            allocate memory & \texttt{pin.initConstraintDynamics(model, data, contact\_models)} \\ 
            forward dynamics with contact constraints & \texttt{pin.constraintDynamics(model, data, q, v, tau, contact\_models, contact\_datas, [prox\_settings])} \\ \hline
            derivatives of the forward dynamics with kinematic constraints & \texttt{pin.computeConstraintDynamicsDerivatives(\newline model, data, contact\_models, contact\_datas, prox\_settings)} \\
        \end{tabular} 
        
    \end{minipage}
};
\node[fancytitle] at (box.north) {Constraint Dynamics};
\end{tikzpicture}
%------------ End Constraint Dynamics  ---------------------------

%------------ Begin Impulse Dynamics -------------------------
% TODO dupliacte with contact dynamics ? not really, different args
\begin{tikzpicture}
\node [mybox] (box){%
    \begin{minipage}{0.9\linewidth}
        \rowcolors{1}{white}{gray!20}
        \begin{tabular}{p{3cm} | p{8.5cm}}
            impulse dynamics with contact constraints & \texttt{pin.impulseDynamics(model, data, q, v, contact\_models, contact\_datas, r\_coeff, mu)} \\ \hline
            impulse dynamics derivatives & \texttt{pin.computeImpulseDynamicsDerivatives(model, data, contact\_models, contact\_datas, r\_coeff, prox\_settings)} \\
            
        \end{tabular} 
        
    \end{minipage}
};
\node[fancytitle] at (box.north) {Impulse Dynamics};
\end{tikzpicture}
%------------ End Impulse Dynamics  ---------------------------


%------------ Begin Cholesky -------------------------
\begin{tikzpicture}
\node [mybox] (box){%
    \begin{minipage}{0.9\linewidth}
        \rowcolors{1}{white}{gray!20}
        \begin{tabular}{p{3cm} | p{8.5cm}}
            Cholesky decomposition of the joint space inertia matrix & \texttt{pin.cholesky.decompose(model, data)} \\
            $x$ of $ M x = y$ & \texttt{pin.cholesky.solve(model, data, v)} \\
            inverse of the joint space inertia matrix & \texttt{pin.cholesky.computeMinv(model, data)} \\
        \end{tabular} 
        
    \end{minipage}
};
\node[fancytitle] at (box.north) {Cholesky};
\end{tikzpicture}
%------------ End Cholesky  ---------------------------


%------------ Begin Viewer --------------
\begin{tikzpicture}
\node [mybox] (box){%
    \begin{minipage}{0.9\linewidth}
        \rowcolors{1}{white}{gray!20}
        \centerline{\textbf{Get started}}
        \begin{tabular}{p{3cm} | p{8.5cm}}
            create viewer & \texttt{mv = pin.visualize.MeshcatVisualizer} \\
            load model & \texttt{viz = mv(model, collision\_model, visual\_model)} \\
            initialize & \texttt{viz.initViewer(loadModel=True)} \\
            display & \texttt{viz.display(q)} \\ 
        \end{tabular} \newline
        

        \centerline{\textbf{Add basic shapes}}
        \begin{tabular}{p{3cm} | p{8.5cm}}
            sphere & \texttt{viz.viewer[name].set\_object(meshcat.geometry.\newline Sphere(size), material)} \\ \hline
            box & \texttt{viz.viewer[name].set\_object(meshcat.geometry.\newline Box([sizex, sizey, sizez]), material)}\\ \hline
        \end{tabular} \newline

        \centerline{\textbf{Display}}
        \begin{tabular}{p{3cm} | p{8.5cm}}
            change placement of geometry [name] & \texttt{viz.viewer[name].set\_transform(\newline meshcat\_transform(xyzquat\_placement))} \\ \hline
            % refresh & \texttt{viz.refresh()}
        \end{tabular} \newline

    \end{minipage}
};

\node[fancytitle, right=10pt] at (box.north west) {Viewer};
\end{tikzpicture}
%------------ End Viewer --------------

\end{multicols*}
\end{document}